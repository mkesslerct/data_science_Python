% 
% Septiembre 2020
% Author: Mathieu Kessler
% Universidad Politécnica de Cartagena
% https://personas.upct.es/perfil/mathieu.kessler
% 
% 
\documentclass[handout,9pt]{beamer}
\definecolor{links}{HTML}{2A1B81}
\hypersetup{colorlinks,linkcolor=,urlcolor=links}
 \usepackage[spanish]{babel}
\usepackage{colortbl}
\usepackage{graphicx}
\usepackage{amsmath,amssymb}
\usepackage{comma}
\usepackage{fancybox,color}
\usepackage[utf8]{inputenc}
\graphicspath{{fig/}}
\setbeamertemplate{navigation symbols}{}
% -----------------------------------------------------------------------------
% To include multple images sequentially
% https://codeyarns.com/2010/02/09/how-to-do-image-animation-in-beamer/
% ------------------------------------------------------------------------------
 \usepackage{xmpmulti}
%------------------------------------------------------------------------------
\usepackage{xcolor}
\definecolor{mycodecolor}{rgb}{0.65,0.25,0.1}
\newcommand{\inlinecode}[1]{{\tt \textcolor{mycodecolor}{#1}}}
\usepackage{pythontex}
%--------------------
\usepackage{beamerthemeshadow}
\usepackage{xmpmulti}
\usepackage{mathtools}
\DeclarePairedDelimiter\abs{\lvert}{\rvert}%
\usepackage{tabularx}
\renewcommand\tabularxcolumn[1]{b{#1}}
\newcommand{\field}[1]{\mathbb{#1}}
\newcommand{\E}{\field{E}}
\newcommand{\R}{\field{R}}
\newcommand{\N}{\field{N}}
\newcommand{\Z}{\field{Z}}
\newcommand{\Q}{\field{Q}}
\newcommand{\EE}{\field{E}}
\newcommand{\FF}{\field{F}}
\newcommand{\GG}{\field{G}}
\renewcommand{\L}{\field{L}}
\renewcommand{\P}{\field{P}}
\newcommand{\LL}{{\mathfrak L}}

% define el folder donde del workspace, para cambiar inglés, ids,
% etc..
\newcommand{\workspacefolder}{stat\_labs }


\begin{document}
\title{Más sobre listas y diccionarios}

\author[Mathieu Kessler]{Mathieu Kessler}
\institute[]{Departamento de Matemática Aplicada y Estadística \\ Universidad Politécnica de Cartagena}
\date{@kessler\_mathieu}{}%{\href{https://code.visualstudio.com}{https://code.visualstudio.com}}
\titlegraphic{\includegraphics[width=3cm]{../figures/python_logo}}

\begin{frame}
  \titlepage
\end{frame}


\begin{frame}[fragile]
  \frametitle{Más sobre listas}
  Vimos que, para acceder a elementos de listas, indexamos su posición
  (con corchetes)
{\footnotesize
  \begin{pyconsole}
my_list = ['data', 'science', 'analytics', 'statistics', 'mathematics']    
print(my_list[1])
\end{pyconsole}
}
\pause
\begin{block}{Cortes (slices)}
A menudo necesitamos extraer elementos consecutivos de una lista. Para
ello, usamos los llamados cortes (\textit{slicing}), which toman la forma:
\begin{pyverbatim}
my_list[start:stop:step]
\end{pyverbatim}
y extrae los elementos desde la posición  {\tt start} hasta la
posición  {\tt
  stop} (excluyendo ésta), con un paso de  {\tt step}.
\begin{itemize}
\item<3-> Si no se especifica, el tercer parámetro  {\tt step} toma el
  valor por defecto igual a 1.\\
  {\footnotesize
  \pyv{my_list[1:3]} devolverá \pyv{['science', 'analytics']}}
\item<4-> Si no se especifica, el primer parámetro {\tt start} toma el
  valor por defecto igual a 0.\\
  {\footnotesize 
    \pyv{my_list[:2]} devuelve \pyv{['data', 'science']}
    }
\item<5-> Si no se especifica el parámetro {\tt stop}, el corte irá
  hasta el fina de la lista.\\
  {\footnotesize
  \pyv{my_list[1:]} devuelve
  \pyv{['science', 'analytics', 'statistics', 'mathematics']}
  }
\end{itemize}
\end{block}
\end{frame}

\begin{frame}[fragile]
  \frametitle{Algunas operaciones con listas}
  \begin{block}{}
    Las listas, junto con los diccionarios, son tipos de datos
    esenciales en Python.
  \end{block} \pause
  Algunos comandos o métodos básicos para listas: \pause
    \begin{pyconsole}
my_list = [2, 1, -0.5, 0.287, 'abc']      
len(my_list) # devuelve el número de elementos
my_list[4] = 'cba' # cambia el elemento en la posición 4
my_list
del my_list[2] # elimina el elemento en la posición 2, modifica la lista.
my_list
my_list.append('joe') # adjunta, modifica la lista.
my_list
my_list.insert(1, 9.2) # inserta en posición 1, modifica la lista.
my_list
\end{pyconsole}

\end{frame}

\begin{frame}[fragile]
  \frametitle{Cómo hacer una copia de una lista}
  \begin{block}{}
  Para hacer una copia de una lista, podríamos pensar en usar el
  operador de asignación  ``='':
  \begin{pyconsole}
my_duplicate = my_list
\end{pyconsole}
\pyconc[copy0]{my_list = [2, 1, -0.5, 0.287, 'abc']}
\pyconc[copy0]{my_duplicate = my_list}
\begin{pyconsole}[copy0]
my_duplicate
\end{pyconsole}
  \end{block}
\pause
Sin embargo, eso implica que {\tt my\_duplicate} que es el mismo
objeto que  {\tt
  my\_list}, refiriendo a la misma localización en memoria: si
modifico una de ellas, la otra resulta modificada también:
\begin{pyconsole}[copy0]
my_duplicate[0] = -1.2
my_duplicate
my_list
\end{pyconsole}
\end{frame}

\begin{frame}[fragile]
  \frametitle{Cómo hacer una copia de una lista}

La solución es usar el método \pyv{copy}, que crea un nuevo objeto de
tipo lista \pause
  \begin{pyconcode}[copy1]
my_list = [2, 1, -0.5, 0.287, 'abc']      
\end{pyconcode}
\begin{pyconsole}[copy1]
my_duplicate = my_list.copy()
my_duplicate[0] = -1.2
my_duplicate
my_list
\end{pyconsole}
\pause
Es también posible hacer una copia usando el operador de corte: \pause
\begin{pyconcode}[copy1]
my_list = [2, 1, -0.5, 0.287, 'abc']      
\end{pyconcode}
\begin{pyconsole}[copy1]
my_duplicate = my_list[:]
my_duplicate[0] = -1.2
my_duplicate
my_list
\end{pyconsole}

\end{frame}

\begin{frame}[fragile]
  \frametitle{Iterar sobre diccionarios}
  \begin{block}{Obtener claves, valores o elementos de un diccionario}
    Los tres métodos \pyv{keys}, \pyv{values} y \pyv{items} permiten
    extraer respectivamente las claves, valores o elementos de un
    diccionario
  \end{block}
    \begin{pyconsole}
personas = {
   'Pedro': 28,
   'María': 21,
   'Marta': 22
}
print(personas.keys())
print(personas.values())
print(personas.items())
    \end{pyconsole}

\end{frame}

\begin{frame}[fragile]
  \frametitle{Iterar sobre diccionarios}
  Podemos iterar sobre un diccionario aprovechando \pyv{values} o
  \pyv{items}:

  \begin{pyconsole}
personas = {
   'Pedro': 28,
   'María': 21,
   'Marta': 22
}
for nombre, edad in personas.items():
    print(f'{nombre} tiene {edad} años')

  \end{pyconsole}

  \end{frame}
  

\begin{frame}[fragile]
  \frametitle{Comprehensiones de listas o diccionarios}
Una característica muy conveniente de Python es la construcción de
``Comprehensiones''de listas o diccionarios (en inglés ``list
comprehension'', ``dictionary comprehension'').

\begin{block}{Comprehensiones}
  Permiten construir listas o diccionarios usando el constructor con
  corchetes, pero aprovechando iteraciones y condiciones en la
  construcción. Las comprehensiones resultan consisas y legibles a la vez.
  
\end{block}
Por ejemplo, si tenemos una lista con las edades de Pedro, María y
Marta:
\begin{pyconsole}
edades = [28, 21, 22]
\end{pyconsole}
Podemos construir una lista con sus edades dentro de 10 años:
\begin{pyconsole}
[edad + 10 for edad in edades]
\end{pyconsole}
Podemos incluso incorporar una consulta a nuestra comprehensión:

\begin{pyconsole}
[edad + 10 for edad in edades if edad < 25]
\end{pyconsole}
  \end{frame}

  \begin{frame}[fragile]
  \frametitle{Comprehensiones de listas o diccionarios}
Podemos proceder de manera similar para diccionarios:

Consideramos de nuevo el diccionario de personas:
\begin{pyconsole}
personas = {
   'Pedro': 28,
   'María': 21,
   'Marta': 22
}
\end{pyconsole}
Podemos construir un nuevo diccionario con sus edades dentro de 10 años:
\begin{pyconsole}
{nombre: (edad + 10) for nombre, edad in personas.items()}
\end{pyconsole}
\begin{block}{Reto:}
 Si consigo del INE la esperanza de vida adicional en años para cada
 de sus edades:
 \begin{pyverbatim}
esperanza_adicional = {
     28: 53.4
     21: 65.6
     22: 64,5
}
 \end{pyverbatim}
Podríais construir un diccionario con la esperanza de vida de  Pedro, María y
Marta?
\end{block}
\end{frame}
  
  
\end{document}

% 
% Septiembre 2020
% Author: Mathieu Kessler
% Universidad Politécnica de Cartagena
% https://personas.upct.es/perfil/mathieu.kessler
% 
% 
\documentclass[9pt]{beamer}
\definecolor{links}{HTML}{2A1B81}
\hypersetup{colorlinks,linkcolor=,urlcolor=links}
\usepackage[spanish]{babel}
\usepackage{colortbl}
\usepackage{graphicx}
\usepackage{amsmath,amssymb}
\usepackage{comma}
\usepackage{fancybox,color}
\usepackage[utf8]{inputenc}
\graphicspath{{../figures/}}
\setbeamertemplate{navigation symbols}{}
% \usepackage[colorlinks=true]{hyperref}
\usepackage{xcolor}
\definecolor{mycodecolor}{rgb}{0.65,0.25,0.1}

\usepackage{listings}
\newcommand{\inlinecode}[2][Python]{\lstinline[language=#1, basicstyle=\color{mycodecolor}]{#2}}
\usepackage{beamerthemeshadow}
\usepackage{xmpmulti}
\usepackage{mathtools}
\DeclarePairedDelimiter\abs{\lvert}{\rvert}%
\usepackage{tabularx}
\renewcommand\tabularxcolumn[1]{b{#1}}
\newcommand{\field}[1]{\mathbb{#1}}
\newcommand{\E}{\field{E}}
\newcommand{\R}{\field{R}}
\newcommand{\N}{\field{N}}
\newcommand{\Z}{\field{Z}}
\newcommand{\Q}{\field{Q}}
\newcommand{\EE}{\field{E}}
\newcommand{\FF}{\field{F}}
\newcommand{\GG}{\field{G}}
\renewcommand{\L}{\field{L}}
\renewcommand{\P}{\field{P}}
\newcommand{\LL}{{\mathfrak L}}


\begin{document}
\lstset{language=Python}
\title{Crear un entorno virtual con  \inlinecode{conda}}

\author[Mathieu Kessler]{Mathieu Kessler}
\institute[]{Departamento de Matemática Aplicada y Estadística \\ Universidad Politécnica de Cartagena}
%% \date{\textcolor{blue}{Granada, 5 de abril de 2019}}
\titlegraphic{\includegraphics[width=6cm]{python_anaconda}}

\begin{frame}
  \titlepage
\end{frame}

\begin{frame}
  \frametitle{Qué es un entorno virtual?}
  \begin{block}{}
    Un entorno virtual permite crear un espacio de trabajo aislado
    para un proyecto. Se realiza para evitar conflictos de versiones de
    paquetes entre diferentes proyectos. Quizás uno de vuestro
    proyecto depende de una versión anterior de un paquete que otro
    proyecto más reciente. Si los dos proyectos se apoyan en el
    entorno {\tt ``base''} de conda, la actualización del paquete para un
    proyecto podría implicar que el otro dejara de funcionar. 
    
  \end{block}\pause

  La solución es crear un entorno virtual para un proyecto, que usará
  su propia versión de Python y de las librerías que necesita.\\
  \pause

  \begin{block}{Entornos virtuales en conda}
    Existen varios gestores de entornos virtuales. Conda tiene el
    suyo propio, la documentación está disponible en el
    \href{https://docs.conda.io/projects/conda/en/latest/user-guide/tasks/manage-environments.html}{documento
      Anaconda sobre gestión entornos.}
    
  \end{block}
\end{frame}
\begin{frame}
  \frametitle{Crear un entorno virtual en conda}
  \begin{block}{Cinco puntos básicos}
    \begin{itemize}
    \item La creación de un entorno virtual se realiza desde la
      consola de Anaconda.
    \item El comando básico para crear un entorno virtual es
      \inlinecode[bash]{conda create --name myenv} donde {\tt myenv}
      es el nombre que habéis escogido para vuestro entorno virtual.
    \item Una vez creado, tenéis que activarlo si queréis trabajar con
      él desde la consola de Anaconda.
    \item En Visual Studio Code, puede asignar un entorno virtual a un
      workspace y todos los ficheros python de este proyecto se
      ejecutarán en este entorno.
    \item Podéis crear un entorno virtual ``bare''  \inlinecode{
        conda create --name myenv python}, activarlo y después instalar los
      paquetes necesarios para vuestro proyecto con  \inlinecode[bash]{conda install},
      o podéis directamente especificar los paquetes requeridos en el
      momento de la creación del entorno.
    \end{itemize}    
  \end{block}\pause
  Vamos a crear el entorno virtual \inlinecode[bash]{ids} para la
  asignatura, con las librerías que necesitaremos y veremos cómo
  asignarlo a nuestro workspace en Visual Studio Code.
\end{frame}
\begin{frame}
  \frametitle{Creando el entorno}
  Vamos a crear el entorno virtual e instalar directamente los
  paquetes relevantes.
  \begin{block}{Instalación de paquetes y su gestión en  conda}
    El gestor de paquetes de conda los descarga desde 
    \href{https://anaconda.org/search}{Anaconda cloud}, donde la
    compañia Anaconda y otras mantienen versiones binarias para
    diferentes sistemas operativos.
  \end{block}\pause
  Vamos a instalar los siguientes paquetes para IDS:\\ \smallskip
  \begin{tabular}[h]{ll}
    $\bullet$ {\tt numpy}&\onslide<3->{Un paquete para la computación científica. \href{https://numpy.org/}{https://numpy.org/}}\\[0.2cm]

                         % \verb{@numpy_team}}\\
    % $\bullet$ {\tt scipy} & \onslide<4->{\parbox{8.5cm}{Rutinas numéricas: integración, interpolación,
    %                         optimización, algebra lineal y estadística
    %                         \href{https://www.scipy.org}{https://www.scipy.org}}}\\[0.3cm]
    
    $\bullet$ {\tt matplotlib} & \onslide<4->{\parbox{8.5cm}{Librería
                                 para crear visualizaciones.
                                 \href{https://matplotlib.org/}{https://matplotlib.org/}}}\\[0.3cm]
    
    $\bullet$ {\tt pandas} & \onslide<5->{\parbox{8.5cm}{Análisis y
                             manipulación de datos \href{https://pandas.pydata.org/}{https://pandas.pydata.org/}                           
                             }}\\[.2cm]
    $\bullet$ {\tt scikit-learn} &
                                   \onslide<6->{\parbox{8.5cm}{Algoritmos
                                   de machine learning \href{https://scikit-learn.org/stable/}{https://scikit-learn.org/stable/}                           
                                   }}\\[.2cm]
    
    $\bullet$ {\tt ipykernel} &\onslide<7->{\parbox{8.5cm}{
                              Proporciona un entorno interactivo para
                                blocs de notas Jupyter.
                              }}
  \end{tabular}\onslide<8->{
  \begin{block}{}
    Es recomendable instalar a la vez  todos los paquetes que pensamos que vamos
    a utilizar. La instalación posterior de un paquete puede dar lugar a
    conflictos de versiones. 
    
  \end{block}
  }
\end{frame}
\begin{frame}
  \frametitle{Creando el entorno}
  Instalaremos también un paquete para el auto-formateo del código.\\ \medskip
  \begin{tabular}[h]{ll}
    $\bullet$ {\tt autopep8} & \onslide<2->{\parbox{8.5cm}{Un
                               formateador que busca cumplir con el
                               estilo de código PEP8}}
  \end{tabular}\\ \bigskip
  \onslide<3->{En la consola de Anaconda, teclead (en una línea)
    \begin{block}{}
      {\tt conda create --name ids -c conda-forge python numpy pandas}\\
      {\tt \hphantom{conda create --name }
        scikit-learn matplotlib ipykernel autopep8}
    \end{block}
  }\onslide<4->{
  Otra opción más sencilla es descargar del Aula virtual el fichero de
  requerimientos ids\_environment.yml y en la consola de Anaconda escribir
  \begin{block}{}
    {\tt conda env create -f ids\_environment.yml}
  \end{block}
}\onslide<5->{
\begin{block}{Para comprobar que se ha creado correctamente}
  En la consola de Anaconda, activad el entorno con {\tt conda
    activate ids} y a continuación, obtened la lista de paquetes con
  {\tt conda list}, comprobando que están los requeridos.
\end{block}
}
\end{frame}
\end{document}

% 
% Septiembre 2020
% Author: Mathieu Kessler
% Universidad Politécnica de Cartagena
% https://personas.upct.es/perfil/mathieu.kessler
% 
% 
\documentclass[9pt]{beamer}
\definecolor{links}{HTML}{2A1B81}
\hypersetup{colorlinks,linkcolor=,urlcolor=links}
 \usepackage[spanish]{babel}
\usepackage{colortbl}
\usepackage{graphicx}
\usepackage{amsmath,amssymb}
\usepackage{comma}
\usepackage{fancybox,color}
\usepackage[utf8]{inputenc}
\graphicspath{{fig/}}
\setbeamertemplate{navigation symbols}{}
%\usepackage[colorlinks=true]{hyperref}
\usepackage{xcolor}
\definecolor{mycodecolor}{rgb}{0.65,0.25,0.1}


\usepackage{beamerthemeshadow}
\usepackage{xmpmulti}
\usepackage{mathtools}
\DeclarePairedDelimiter\abs{\lvert}{\rvert}%
\usepackage{tabularx}
\renewcommand\tabularxcolumn[1]{b{#1}}
\newcommand{\field}[1]{\mathbb{#1}}
\newcommand{\E}{\field{E}}
\newcommand{\R}{\field{R}}
\newcommand{\N}{\field{N}}
\newcommand{\Z}{\field{Z}}
\newcommand{\Q}{\field{Q}}
\newcommand{\EE}{\field{E}}
\newcommand{\FF}{\field{F}}
\newcommand{\GG}{\field{G}}
\renewcommand{\L}{\field{L}}
\renewcommand{\P}{\field{P}}
\newcommand{\LL}{{\mathfrak L}}

% define el folder donde del workspace, para cambiar inglés, ids,
% etc..
\newcommand{\workspacefolder}{stat\_labs }


\begin{document}
\title{Customizing (a little) Visual Studio Code}

\author[Mathieu Kessler]{Mathieu Kessler}
\institute[]{Departamento de Matemática Aplicada y Estadística \\ Universidad Politécnica de Cartagena}
\date{\href{https://code.visualstudio.com}{https://code.visualstudio.com}}
\titlegraphic{\includegraphics[width=3cm]{../figures/visual_studio_code_logo}}

\begin{frame}
  \titlepage
\end{frame}

\begin{frame}
  \frametitle{Let us begin by changing color theme and the icons
    theme}
  Using the Command Palette (Ctrl-Shift-P), you choose Preferences:
  Color Theme.
  \begin{center}
    \includegraphics[width=6cm]{../figures/color_theme_01}
  \end{center}\pause
  You can choose from a list of preinstalled themes, or decide to
  install additional color themes.
  \begin{center}
    \includegraphics[width=6cm]{../figures/color_theme_015}
  \end{center}\pause
\end{frame}

\begin{frame}
  \frametitle{Let us begin by changing color theme and the icons
    theme}
  If you choose to install additional color themes, it takes you to
  the extensions market place, with the results filtered with
  category: themes.
  \begin{center}
    \includegraphics[width=6cm]{../figures/color_theme_016}
  \end{center}\pause
\end{frame}

\begin{frame}
  \frametitle{Let us begin by changing color theme and the icons
    theme}
  I like the Predawn theme and the Ayu theme for icons. I search for them in the extension marketplace:
  \begin{center}
    \includegraphics[width=8cm]{../figures/predawn_ayu}
  \end{center}\pause
  I first install predawn, and click on ``Select Color Theme''.
   \begin{center}
    \includegraphics[width=6cm]{../figures/predawn_installed}
  \end{center}\pause
  And after installing Ayu, I only click on ``Set File Icon Theme''.
  \begin{center}
    \includegraphics[width=6cm]{../figures/ayu_installed}
  \end{center}
\end{frame}



\begin{frame}
  \frametitle{Visual Studio Code is highly customizable}
  \begin{block}{}There are many many aspects of Visual studio Code that are
  configurable. These are controlled through the values of variables
  that are stored in json files with the name of settings.json.
  \end{block}\pause
  To visualize and modify the  settings, you click the wheel in the bottom left of
  the editor.
  \
  \begin{center}
    \includegraphics[width=4cm]{../figures/settings_01}
  \end{center}
\end{frame}
\begin{frame}
  \frametitle{Visual Studio Code is highly customizable}
  You then have the possibility to search for settings. Notice the three
  tabs, ``User'', ``Workspace'' or ``Folder'', to specify the context to
  which the modified settings should apply.
  \begin{center}
    \includegraphics[width=0.9\textwidth]{../figures/settings_02}
  \end{center}\pause
  \begin{block}{This is the ``ui'' mode of editing settings} We could also see
    all the settings variables in json form (key: value)
  \end{block}
\end{frame}
\begin{frame}
  \frametitle{Visual Studio Code is highly customizable}
  To switch to the ``json'' mode of editing settings, click on the
  upper right icon
  \begin{center}
    \includegraphics[width=0.9\textwidth]{../figures/settings_03}
  \end{center}
\end{frame}
\begin{frame}
  \frametitle{Visual Studio Code is highly customizable}
  \begin{overlayarea}{\textwidth}{\textheight}
    The format if key: value, where the key is the name of each setting variable.
    \begin{center}
      \includegraphics[width=0.7\textwidth]{../figures/settings_04}
    \end{center}\pause
    
    If you hover over a variable, you get a description and a pen icon
    appears which allows you to edit its value.
  \end{overlayarea}
\end{frame}
\begin{frame}
  \frametitle{Visual Studio Code is highly customizable}
    \begin{overlayarea}{\textwidth}{\textheight}
      We can see the two variables that reflect our selection of icons and
      color themes:
      \begin{center}
        \includegraphics[width=0.7\textwidth]{../figures/settings_05}
      \end{center}
  \end{overlayarea}
\end{frame}
\begin{frame}
  \frametitle{Visual Studio Code is highly customizable}
  \begin{overlayarea}{\textwidth}{\textheight}
    You can see that you are in fact editing the user-level {\tt
      settings.json} file:
    \begin{center}
      \includegraphics[width=0.7\textwidth]{../figures/settings_06}
    \end{center}
  \end{overlayarea}
\end{frame}
\begin{frame}
  \frametitle{Visual Studio Code is highly customizable}
  \begin{overlayarea}{\textwidth}{\textheight}
    If you want to see all the default Settings, type ``Open Default
    Settings'' in the Palette Command (Ctrl-Shift-P) 
    \begin{center}
      \includegraphics[width=0.7\textwidth]{../figures/default_settings}
    \end{center}
    It is a huge list of key:value pairs, but you may find  it useful to search for
    a variable, read the description, and copy, paste it and edit it  in your own
    settings.json.
  \end{overlayarea}
\end{frame}
\begin{frame}
  \frametitle{Enable linting}
  \begin{block}{What is linting?}
    Linting gives on the fly hints about ill written code, by marking
  posible mistakes. 
  \end{block}
  To enable it, use the Command Palette and type ``Run Linting'', a
  dialog box will pop up, choose ``Install''.
  \begin{center}
    \includegraphics[width=0.7\textwidth]{../figures/dialog_box_linter}
  \end{center}\pause
  When asked, always use Conda to install the packages.\pause
  \begin{block}{Note: if you are using a virtual environment (for example
      with Miniconda)}
    Pylint will install in the virtual environment, which means you
    will have to install it again if you change environment.
  \end{block}
\end{frame}
\begin{frame}
  \frametitle{Install a formatter}
  \begin{block}{Style guide}
    We will try to be consistent with the accepted guidelines about
    code formating, to improve readibility.
  \end{block}
  \pause
  \begin{quote}
    Good coding style is like correct punctuation: you can manage without it, butitsuremakesthingseasiertoread.\\
    Hadley Wickham.
  \end{quote}
  \pause
  One of the \href{https://www.python.org/dev/peps/}{Python
    Enhancement Proposals}  documents the Style guide for Python
  Code. It is the famous PEP8 document
  \href{https://www.python.org/dev/peps/pep-0008/}{https://www.python.org/dev/peps/pep-0008/}\\
  It is a recommended reading.\pause \\ \medskip
  \begin{block}{Code formatters}
    Code formatters exist that help converting badly formatted code
    into more readable one, some of them targeting the compliance with PEP8.
  \end{block}
\end{frame}
\begin{frame}
  \frametitle{Install a formatter in Visual Studio Code}
  Open a {\tt .py} file or create one, and in the Command Palette ,
  type ``Format document''
  \begin{center}
    \includegraphics[width=0.7\textwidth]{../figures/format_document_01}
  \end{center}
  A dialog box pops up:
  \begin{center}
    \includegraphics[width=0.7\textwidth]{../figures/dialog_box_format}
  \end{center}
  You can say yes to install autopep8, or you may choose either black
  or yapf. (When asked, use Conda to install packages)\pause
  \begin{block}{}
    Remember to format your files from time to time with {\tt Format
      Document} or Shift + Alt + F
  \end{block}

  
\end{frame}
\end{document}
